\documentclass[12pt]{article}
\usepackage{graphicx}
\usepackage{float}
\usepackage{url}
\usepackage{hyperref}
\usepackage{subcaption}
\usepackage[style=numeric,sorting=none,maxbibnames=9,autopunct=true,babel=hyphen,hyperref=true,backend=biber]{biblatex}

\bibliography{references}
\begin{document}

\title{Cosmology with Galaxy Clusters \\ Cosmology Project}
\author{Carlos Rafael Salazar Letona \\ s21205751}
\maketitle

\section{Introduction}


\subsection{$\Lambda$CDM model}
The $\Lambda$CDM model is considered the standard model of cosmology, as it is the simplest theory that reproduces the most important results, such as \cite{ref:lambdacdm}:

\begin{enumerate}
\item the existence and structure of the CMB
\item the large-scale structure in the distribution of galaxies
\item the observed abundances of the lightest atoms
\item the accelerating expansion of the universe
\end{enumerate}
In this model, the energy of the universe is divided in three components: dark energy, cold dark matter, and ordinary matter.


\subsection{Power Spectrum}
The power spectrum is used to describe the density contrast of the universe, and it is calculated using the Fourier transform of the matter correlation function\cite{ref:powerspectrum}.


\section{Cosmological distances}

\subsection{}



\section{Flux and observable range of XXL survey}



\section{Mass variance and non-linear mass scale}



\section{Number counts of haloes}



\section{Discussion}


\begin{figure}[htp]
	\centering
	\begin{subfigure}[b]{0.49\textwidth}
		\centering
		\includegraphics[width=\textwidth]{Graphs/measureda05.png}
	\end{subfigure}
	\hfill
	\begin{subfigure}[b]{0.49\textwidth}
		\centering
		\includegraphics[width=\textwidth]{Graphs/theoreticala05.png}
	\end{subfigure}
	\caption{Measured and theoretical power spectrums}
	\label{fig:spectrum05}
\end{figure}

\begin{figure}[htp]
	\centering
	\begin{subfigure}[b]{0.49\textwidth}
		\centering
		\includegraphics[width=\textwidth]{Graphs/densityTimeOm0032Long.png}
	\end{subfigure}
	\hfill
	\begin{subfigure}[b]{0.49\textwidth}
		\centering
		\includegraphics[width=\textwidth]{Graphs/densityTimeOm032Long.png}
	\end{subfigure}
	\begin{subfigure}[b]{0.49\textwidth}
		\centering
		\includegraphics[width=\textwidth]{Graphs/densityTimeOm09Long.png}
	\end{subfigure}
	\caption{Densities at different scale factors for different $\Omega_{m, 0}$ values.}
	\label{fig:densities}
\end{figure}

\begin{figure}[htp]
	\centering
	\begin{subfigure}[b]{0.49\textwidth}
		\centering
		\includegraphics[width=\textwidth]{Graphs/densityTimeOm09Long.png}
	\end{subfigure}
	\hfill
	\begin{subfigure}[b]{0.49\textwidth}
		\centering
		\includegraphics[width=\textwidth]{Graphs/densityTimeNonRandom.png}
	\end{subfigure}
	\caption{Simulation with and without addition of displacement field.}
	\label{fig:randomField}
\end{figure}



\section*{Conclusion}



\section*{References}
\printbibliography[heading=none]

\end{document}